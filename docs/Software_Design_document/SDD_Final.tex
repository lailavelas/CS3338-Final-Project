\documentclass{article}
\usepackage{geometry}
\geometry{a4paper, margin=1in}
\usepackage{hyperref}
\usepackage{longtable}

\title{Software Design Document (SDD) - Machine Learning Platform for Intelligent Water Systsems Management - Final Project}
\author{Robert Castro\\
Calvin Chau\\
Yvan Michel Kemsseu Yobeu\\
Laila Velasquez\\
Kassandra Vera}
\date{Friday, May 9, 2025}

\begin{document}
\maketitle
\newpage
\tableofcontents

\newpage

\section*{Revision History}
\begin{center}
\begin{tabular}{|l|l|l|l|}
\hline
Name & Date & Reason For Changes & Version \\
\hline
Laila Velasquez & 5/5 & 1.0-2.5 & 1.0 \\
\hline
Laila Velasquez & 5/7 & 3.0-4.0 & 1.1 \\
\hline
 &  &  &  \\
\hline
 &  &  &  \\
\hline
 &  &  &  \\
\hline
 &  &  &  \\
\hline
 &  &  &  \\
\hline
 &  &  &  \\
\hline
 &  &  &  \\
\hline
\end{tabular}
\end{center}
\newpage

\section{Introduction}
\subsection{Purpose}
The purpose of this Software Design Document (SDD) is to provide a detailed architecture and design overview of the Intelligent Water Systems Management Platform. This platform is designed to optimize water usage, monitor real-time water consumption, and predict maintenance needs through integrated sensor networks and data analytics.

\subsection{Intended Audience}
This document is intended for:
\begin{itemize}
\item Software developers implementing the platform
\item Project managers overseeing development
\item Stakeholders evaluating project progress
\item Quality assurance teams for testing and validation
\end{itemize}

\subsection{Overview}
The Intelligent Water Systems Management Platform leverages real-time data from water sensors to track consumption, detect leaks, and optimize usage patterns. It is designed for municipalities, industrial complexes, and residential areas to achieve efficient water management.

\section{System Architecture}
\subsection{Workflow}
\begin{itemize}
\item Data Collection → Sensor Data Stream
\item Data Processing → Real-time Monitoring \& Alerts
\item Data Storage → Secure Cloud Database
\item Data Visualization → Web Application Dashboard
\end{itemize}

\subsection{Site Breakdown}
\begin{itemize}
\item \textbf{Home Dashboard} - Overview of current water usage, alerts, and predictions.
\item \textbf{Analytics Page} - Visualization of consumption trends and historical data.
\item \textbf{Settings} - Configuration options for sensors and user preferences.
\item \textbf{Reports} - Generation of monthly, weekly, and daily reports for users.
\end{itemize}

\section{User Interface}
\subsection{How to Use}
Users log in to the dashboard to view real-time data, configure settings, and download usage reports. Alerts are displayed prominently for immediate action.

\subsection{Database Explanation}
Data from sensors is stored in a cloud-based SQL database, allowing for efficient querying and analysis. The structure follows normalized principles for fast data access and integrity.

\section{Glossary}
\begin{itemize}
\item \textbf{Sensor Network} - A collection of interconnected sensors that track water usage.
\item \textbf{Dashboard} - The user interface for monitoring and managing water data.
\item \textbf{SQL Database} - A structured database for storing sensor data securely.
\item \textbf{Real-time Monitoring} - The capability to view water consumption as it happens.
\end{itemize}

\section{References}
\begin{itemize}
\item Project Source: \url{[https://ascent.cysun.org/project/project/view/216}
\item Development Tools: Docker, Flask, SQL, HTML/CSS
\item Documentation Standards: IEEE 1016-2009
\end{itemize}

\end{document}
