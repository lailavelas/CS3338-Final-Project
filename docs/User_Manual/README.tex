\documentclass{article}
\usepackage{geometry}
\geometry{a4paper, margin=1in}
\usepackage{hyperref}
\usepackage{longtable}
\title{User Manual - Machine Learning Platform for Intelligent Water Systems Management - Final Project}
\author{Robert Castro\\
Calvin Chau\\
Yvan Michel Kemsseu Yobeu\\
Laila Velasquez\\
Kassandra Vera}
\date{Friday, May 9, 2025}

\begin{document}
\maketitle
\pagebreak
\tableofcontents
\pagebreak

\section{Introduction}
\subsection{Purpose}
This User Manual provides step-by-step guidance for users to effectively navigate and utilize the Intelligent Water Systems Management Platform.

\subsection{Overview}
The platform allows users to monitor water consumption, detect anomalies, and optimize water usage through real-time data visualization and predictive analytics.

\section{Jira Link}
You can access the Jira board for project management and sprint planning here: \url{https://csula-kv.atlassian.net/jira/software/projects/WSM/boards/6}

\section{Formal Objective Breakdown}
The primary objectives of this platform are:
\begin{itemize}
\item Real-time monitoring of water consumption
\item Automated alerts for leak detection
\item Predictive analysis for maintenance
\item Detailed analytics for optimizing water usage
\end{itemize}

\section{Goals and Importance}
This platform is designed to bridge the gap in water consumption monitoring by providing:
\begin{itemize}
\item Enhanced water usage efficiency
\item Early detection of leaks to prevent waste
\item Data-driven insights for better resource management
\end{itemize}
By offering real-time monitoring and predictive analytics, it empowers municipalities and property owners to optimize water consumption and reduce waste.

\section{Installation and Setup}
\subsection{System Requirements}
\begin{itemize}
\item Docker (latest version)
\item Node.js (for frontend)
\item SQL Database
\end{itemize}

\subsection{Installation Steps}
\begin{enumerate}
\item Clone the repository from GitHub.
\item Navigate to the project directory.
\item Run \texttt{docker-compose up} to initialize all services.
\item Open the application at \url{http://localhost:5000}.
\end{enumerate}

\subsection{Docker Configuration}
The Docker setup includes:
\begin{itemize}
\item \textbf{Frontend}: React-based dashboard
\item \textbf{Backend}: Flask for API management
\item \textbf{Database}: SQL instance for data storage
\end{itemize}

\section{User Interface Guide}
\subsection{Dashboard Overview}
The main dashboard displays:
\begin{itemize}
\item Real-time water consumption data
\item Alerts for abnormal usage
\item System health status
\end{itemize}

\subsection{Analytics Page}
\begin{itemize}
\item Graphical representation of historical data
\item Comparative analysis of water usage trends
\end{itemize}

\subsection{Settings}
\begin{itemize}
\item Sensor configurations
\item Alert preferences
\end{itemize}

\subsection{Reports}
\begin{itemize}
\item Generate and export reports in CSV or PDF formats
\end{itemize}

\section{Usage Instructions}
\subsection{Logging In}
\begin{enumerate}
\item Open the application at \url{http://localhost:5000}.
\item Enter your username and password.
\item Click \textbf{Login}.
\end{enumerate}

\subsection{Viewing Data}
Navigate to the \textbf{Dashboard} to see live metrics.

\subsection{Configuring Alerts}
Go to \textbf{Settings} and enable notifications for high usage or leaks.

\subsection{Generating Reports}
Click on \textbf{Reports} and select the time range for data.

\section{Troubleshooting}
\subsection{Common Issues}
\begin{itemize}
\item \textbf{Docker container not starting} → Verify Docker installation and run \texttt{docker-compose up --build}.
\item \textbf{Dashboard not loading} → Clear browser cache and retry.
\end{itemize}

\subsection{Error Messages}
\begin{itemize}
\item \textbf{500 Internal Server Error} → Restart Docker containers.
\item \textbf{404 Not Found} → Check Docker logs for missing dependencies.
\end{itemize}

\section{Appendix}
\subsection{API Documentation}
Refer to the API endpoints for detailed usage.

\subsection{Database Schema}
SQL-based with normalized tables for optimized querying.

\end{document}
