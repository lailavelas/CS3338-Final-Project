\documentclass{article}
\usepackage{geometry}
\geometry{a4paper, margin=1in}
\usepackage{hyperref}
\usepackage{longtable}
\title{Design Specifications - Machine Learning Platform for Intelligent Water Systems Management - Final Project}
\author{Robert Castro\\
Calvin Chau\\
Yvan Michel Kemsseu Yobeu\\
Laila Velasquez\\
Kassandra Vera}
\date{Friday, May 9, 2025}

\begin{document}
\maketitle
\pagebreak

\tableofcontents
\newpage

\section{Introduction}
\subsection{Purpose}
The Design Specifications document provides a comprehensive breakdown of the architecture, components, and features of the Intelligent Water Systems Management Platform. It is intended to serve as a reference for developers and stakeholders during implementation and testing phases.

\subsection{Scope}
This document covers the design and functional components of the platform, including database architecture, API specifications, and user interface design.

\subsection{Design Goals}
\begin{itemize}
\item Optimize water usage through real-time monitoring
\item Provide predictive analytics for maintenance
\item Enhance user experience with intuitive dashboards
\end{itemize}

\section{System Design Overview}
\subsection{Architecture Design}
The platform is built on a client-server model using Flask for backend APIs and SQL for database management.

\subsection{Component Design}
\begin{itemize}
\item \textbf{Frontend}: HTML, CSS, JavaScript for interactive dashboards
\item \textbf{Backend}: Flask for routing and API management
\item \textbf{Database}: SQL for structured data storage
\item \textbf{Deployment}: Docker for containerized services
\end{itemize}

\section{Breakdown of Pages and Components}
\subsection{Home Dashboard}
Displays real-time water usage, health status, and alerts.

\subsection{Analytics Page}
Provides historical data visualization and comparative analysis.

\subsection{Settings}
Allows users to configure sensors, set thresholds for alerts, and manage account settings.

\subsection{Reports}
Generates detailed water consumption reports in PDF and CSV format.

\subsection{API Endpoints}
\begin{itemize}
\item \textbf{GET /api/v1/water-usage} - Retrieves real-time water usage data.
\item \textbf{POST /api/v1/alerts} - Sets alert thresholds for leak detection.
\item \textbf{GET /api/v1/reports} - Downloads usage reports.
\end{itemize}

\section{Functional Requirements}
\subsection{Sensor Data Ingestion}
Real-time collection of water usage data.

\subsection{Real-time Monitoring}
Display live data on dashboards.

\subsection{Alerts and Notifications}
Automated alerts for anomalies (e.g., leaks).

\subsection{Data Visualization}
Graphical representation of water consumption.

\section{Non-Functional Requirements}
\begin{itemize}
\item \textbf{Performance}: Capable of handling large datasets
\item \textbf{Scalability}: Expandable to multiple regions
\item \textbf{Security}: Encrypted data storage and secure API communication
\end{itemize}

\section{Technical Specifications}
\subsection{Database Schema}
Optimized for fast querying and retrieval.

\subsection{API Endpoints}
RESTful design for sensor data, user info, and reporting.

\subsection{Docker and Flask Integration}
Containerized services for scalability.

\section{UI/UX Design Specifications}
\subsection{Dashboard Layout}
Intuitive and user-friendly.

\subsection{User Interactions}
Real-time updates and interactive graphs.

\section{Testing and Validation}
\begin{itemize}
\item \textbf{Unit Testing}: For individual components
\item \textbf{Integration Testing}: For combined modules
\item \textbf{Performance Testing}: For large-scale usage
\end{itemize}

\section{Deployment Strategy}
\subsection{Docker Configuration}
Defined services for frontend, backend, and database.

\subsection{Cloud Hosting Setup}
For scalability and accessibility.

\end{document}
